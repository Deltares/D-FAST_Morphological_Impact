\chapter{Code structure}

%\section{Introduction}

\dfastmi code is subdivided into 5 files:

\begin{itemize}
\item \file{\_\_init\_\_.py} module level file containing mainly the version number.
\item \file{\_\_main\_\_.py} module level file containing argument parsing and call to \keyw{dfastmi.cmd.run()}.
\item \file{cmd.py} containing the main run routine.
\item \file{cli.py} for the backward compatible interactive command line interface (equivalent to WAQMORF).
\item \file{resources.py} is used to link to the \dfastmi logo.
\end{itemize}

and 5 sub-packages:

\begin{itemize}
\item \file{gui} contains all functionality related to the graphical user interface.
\item \file{batch} implements the workflow for the actual \dfastmi analysis (triggered either by a \keyw{-{}-mode batch} call or by clicking the \button{Compute} button in the graphical user interface).
\item \file{config} implements the processing and checking of the river and analysis configuration files.
\item \file{io} contains the reading, parsing and writing of all configuration and result files.
\item \file{kernel} contains the scientific steps of the analysis.
\end{itemize}

The detailed description of the code per class and routine is beyond the scope of this document.
For more details the reader is referred to the documentation strings included in the source code.
