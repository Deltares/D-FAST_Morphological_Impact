\chapter{Conclusion}

The cases reported in \autoref{Chp:Backward} show that \dfmi version 3 gives the same results as earlier versions of \dfmi when run in backward compatibility mode using the same (WAQUA) input data.
The cases studied in \autoref{Chp:Morphology} show that for moderate interventions (big enough to have substantial influence, but small enough to not exceed the validity assumptions of the method) \dfmi version 3 gives results that match well with the long-term morphological impact as determined by full morphological simulations especially with respect to the along-stream (1D) variability.
The cases studied in \autoref{Chp:Verschil} show that the results of the new approach, implemented in \dfmi version 3, are in line with the results of the previous methodology implemented by WAQMORF and \dfmi version 2.
The new approach has the benefit that it removes the variability in the flow conditions that need to be simulated.

It should be noted that all analyses have been carried out using results obtained from either Delft3D 4 or WAQUA.
However, \dfmi 3 is intended for use with results obtained from \dflowfm.
So, can we, based on these cases, conclude anything about the validaty of \dfmi 3 when used in combination with \dflowfm results?
During the initial development of \dfmi 2 it was verified that the 3-discharge method gave similar results when using WAQUA results and when using \dflowfm results \citep{DFAST2020}.
This is as expected since both models have been calibrated in a similar way against the same data sets.
Therefore, it is expected that differences will be mainly observed when studying small interventions that are barely resolved on the computational mesh.
This conclusion also holds for the new \emph{stepped-hydrograph} method implemented in \dfmi 3.
Hence, the validation testing reported here is also relevant for the use of \dfmi 3 in combination with \dflowfm results.

The overall conclusion is that the results of \dfmi version 3 are reliable (valid) under the conditions described in chapter 3 (User Guidance) of the \citet{um}.
