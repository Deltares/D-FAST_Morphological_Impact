\chapter{Introduction}
In 2020 the rapid assessment tool \dfastmi (MI) was developed for the Rijkswaterstaat 6th generation river models based on \dhydrosuite / \dflowfmfull (FM).
The algorithms implemented in \dfmi version 2 correspond to the algorithms implemented in the former, equivalent tool WAQMORF included in the SIMONA Suite.
In 2021 a number of extensions to the \dfastmi algorithms were implemented in a pilot version and evaluated.
\dfmi version 3, the current release, is based on the first of these extensions, namely the \emph{stepped-hydrograph} approach for reaches with dominant river discharge forcing.

This document describes the cases used to validate this version.
It is subdivided into three chapters:

\begin{itemize}
\item \nameref{Chp:Backward}. This chapter verifies that results of \dfmi 3, when running in the classic 3-discharge backward compatibility mode, are identical to the results obtained using \dfmi 2 and WAQMORF.
This validates that the results of the backward compatibility mode are correct, as the old results have already been accepted as realistic in previous studies.
\item \nameref{Chp:Morphology}. This chapter compares the results of \dfmi 3, when using the new \emph{stepped-hydrograph} mode, with the results of actual long-term morphological simulations.
This validates that the results of the new \emph{stepped-hydrograph} mode agree with those obtained using long-term morphological simulations for a number of reference interventions.
Please note that this does not guarantee that results will agree for all possible interventions. 
\item \nameref{Chp:Verschil}. This chapter compares the results of \dfmi 3, when using the new \emph{stepped-hydrograph} mode, with the results obtained using the classic 3-discharge approach.
This verifies that the results of the new \emph{stepped-hydrograph} approach do not deviate substantially from results obtained using the old method.
Note that the old results were accepted as realistic in previous studies.
\end{itemize}
