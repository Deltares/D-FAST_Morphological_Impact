\chapter{File formats}

The software distinguishes 6 files:

\begin{itemize}
\item The \emph{rivers configuration file} defines the branches and reaches, and all parameter settings specific for the overall system, per branch or per reach.
\item The \emph{dialog text file} defines all strings to be used in the interaction with the users (GUI, report, or error messages).
\item The \emph{analysis configuration file} defines the settings that are relevant for a specific execution of the algorithm, i.e.~for a specific branch, reach and measure.
\item The \emph{simulation result files} define the spatial variations in the velocities and water depths as needed by the algorithm.
\item The \emph{report file} contains a logging of the settings and lumped results for the analysis.
\item The \emph{spatial output file} contains the estimate of the spatial variation in the sedimentation and erosion patterns that will result from the measure (minimum, mean and maximum).
\end{itemize}

Each file type is addressed separately in the following subsections.

\section{rivers configuration file}

The rivers configuration file follows the common ini-file format.
The file must contain a \keyw{[General]} block with a keyword \keyw{Version} to indicate the version number of the file.
The initial version number will be \keyw{1.0}.

Besides the \keyw{[General]} block version 1.0 files should only contain data blocks for the river branches (in Dutch: takken).
The names of those blocks will be used as branch names.
For instance, a block [My branch] will be processed as a branch called "My branch".
The order of the branches will correspond to the order of the blocks in the file.
The branch block defines the reaches (in Dutch: stukken) to be distinguished as well as the branch or reach specific parameter settings.
The \keyw{[General]} block may contain default values for the 
Further details follow below.

\begin{tabular}{l|l|p{8cm}}
Block & Keyword & Description \\ \hline
\keyw{General} & \keyw{Version} & Version number. Must be \keyw{1.0} \\
BranchName<i> & \keyw{Reach<j>} & Name of reach <j> within branch <i> \\
BranchName<i> & \keyw{QLocation} & Location at which discharges for branch <i> are defined \\
\keyw{*} & \keyw{QStagnant} & Discharge \unitbrackets{m\textsuperscript{3}/s} below which main channel flow can be assumed stagnant \\
\keyw{*} & \keyw{QMin} & Minimum discharge \unitbrackets{m\textsuperscript{3}/s} at which measure becomes active \\
\keyw{*} & \keyw{QFit} & Two discharges \unitbrackets{m\textsuperscript{3}/s} used for representing the exceedance curve \\
\keyw{*} & \keyw{QLevels} & Four characteristic discharges \unitbrackets{m\textsuperscript{3}/s} used by algorithm \\
\keyw{*} & \keyw{dQ} & Two discharge adjustments \unitbrackets{m\textsuperscript{3}/s} used by algorithm \\
\keyw{*} & \keyw{NWidth} & Normal width \unitbrackets{m} of main channel \\
\keyw{*} & \keyw{PRLow} & Low flow propagation rate \unitbrackets{km/yr} \\
\keyw{*} & \keyw{PRHigh} & High flow propagation rate \unitbrackets{km/yr} \\
\keyw{*} & \keyw{UCrit} & Critical (minimum) velocity \unitbrackets{m/s} for sediment transport
\end{tabular}

The second value of \keyw{QLevels} corresponds to the typical bankfull discharge.
All keywords listed for block \keyw{*} may occur either in the \keyw{[General]} block or in one of the branch specific blocks where they may optionally be concatenated with the reach number <j>.
Those keywords may thus occur in three flavours:

\begin{enumerate}
\item Plain keyword in block \keyw{[General]}: global default value valid for all branches
\item Plain keyword in branch specific block: default value for that branch (overrules any global default)
\item Keyword followed by reach number <j> in branch specific block: value valid for that reach on that branch.
\end{enumerate}

\subsubsection*{Example}

The following excerpt of the default \keyw{Dutch\_rivers.ini} configuration file shows the \keyw{[General]} as well as the first part of the \keyw{[Bovenrijn \& Waal]} block for the first branch.
It includes a global value of 0.3 for \keyw{UCrit} and 100 for \keyw{QMin}.
The other parameters are specified at branch level and mostly uniform for the whole branch.
Only \keyw{NWidth} and \keyw{PRLow} vary depending on the reach selected.

\begin{Verbatim}
    [General]
        Version    = 1.0

        UCrit      = 0.3
        QMin       = 1000

    [Bovenrijn & Waal]
        QLocation  = Lobith
        QStagnant  = 800
        QFit       = 800  1280
        QLevels    = 3000  4000  6000  10000
        dQ         = 1000  1000
        PRHigh     = 3.65
        
        Reach1     = Bovenrijn                    km  859-867
        NWidth1    = 340
        PRLow1     = 0.89
        
        Reach2     = Boven-Waal                   km  868-886
        NWidth2    = 260
        PRLow2     = 0.81

        ... continued ...
\end{Verbatim}

\section{dialog text file}

The dialog text file uses the block labels enclosed by square brackets of the common ini-file format, but the lines in between the blocks are treated verbatim and don't list keyword/value pairs.
Every print statement in the program is associated with a short descriptive identifier.
These identifiers show up in the dialog text file as the block labels.
The text that follows the block label will be used at that location in the program.
The order of the blocks in the file is not important.
Please note that every line is used as is, so don't add indentations or blank lines unless you want those to show up during the program execution.
Most blocks may contain any number of lines, but some blocks may only contain a single line in particular block that start with \keyw{gui\_} or \keyw{filename\_}.
Some data blocks may contain one or more named placeholders, e.g. \keyw{{version}}, used for inserting values by means of the Python \keyw{format()} method.

\subsubsection*{Example}

The following excerpt of the default \keyw{messages.NL.cfg} dialog text file shows the string definition for 5 identifiers, namely '' (the identifier for an empty line), 'header', 'confirm', 'confirm\_or' and 'confirm\_or\_restart'.
The header string contains one placeholder, namely \keyw{{version}} for the the version number.

\begin{Verbatim}
    []
    
    [header]
    Dit programma is de "WAQUA vuistregel" voor het schatten
    van lokale morfologische effecten door een lokale ingreep
    (zie RWS-WD memo WAQUA vuistregel 20-10-08).
    
    Het is een inschatting van evenwichtsbodemveranderingen
    in het zomerbed die als gevolg van een ingreep en zonder
    aangepast beheer na lange tijd ontstaan.
    
    Dit betreft het effect op de bodem in [m]:
    
        jaargemiddeld zonder baggeren
        maximaal (na hoogwater) zonder baggeren
        minimaal (na laagwater) zonder baggeren
    
    Met deze bodemveranderingen kunnen knelpunten worden
    gesignaleerd. De resultaten zijn niet direkt geschikt
    voor het bepalen van de invloed op vaargeulonderhoud!
    
    De jaarlijkse sedimentvracht door de rivier bepaalt de
    termijn waarbinnen dit evenwichtseffect kan ontwikkelen.'
    
    
    Dit is versie {version}.
    
    [confirm]
    Bevestig met "j" ...
    
    [confirm_or]
    Bevestig met "j", anders antwoord met "n" ...
    
    [confirm_or_restart]
    Bevestig met "j", of begin opnieuw met "n" ...
    
    ... continued ...
\end{Verbatim}

\section{analysis configuration file}

The analysis configuration file follows the common ini-file format.
The file must contain a \keyw{[General]} block with a keyword \keyw{Version} to indicate the version number of the file.
The initial version number will be \keyw{1.0}.

Version 1.0 files must contain in the \keyw{[General]} block also the keywords \keyw{Branch} and \keyw{Reach} to identify the branch (in Dutch: tak) and reach (in Dutch: stuk) in which the measure is located.
The specified names may be shortened, but they should uniquely identify the branch and reach amongst the names of the other branches and reaches.
Optionally, the same block may also contain \keyw{Qmin}, \keyw{QBankfull} and \keyw{UCrit} values representative for this particular measure if they differ from those typical for the selected reach.
These items are sufficient for a basic analysis.
For a full spatial analysis the user needs to specify the names of the D-Flow FM map-files containing the results of the simulations without measure (reference) and with measure for the selected discharges Q1, Q2, and Q3.

\begin{tabular}{l|l|p{8cm}}
Block & Keyword & Description \\ \hline
\keyw{General} & \keyw{Version} & Version number. Must be \keyw{1.0} \\
\keyw{General} & \keyw{Mode} & \keyw{WAQUA export} or \keyw{D-Flow FM map} (the latter is the default) \\
\keyw{General} & \keyw{Branch} & Name of the selected branch \\
\keyw{General} & \keyw{Reach} & Name of the selected reach \\
\keyw{General} & \keyw{QMin} & Minimum discharge \unitbrackets{m\textsuperscript{3}/s} at which measure becomes active \\
\keyw{General} & \keyw{QBankfull} & Discharge \unitbrackets{m\textsuperscript{3}/s} at which measure reaches bankfull \\
\keyw{General} & \keyw{UCrit} & Critical (minimum) velocity \unitbrackets{m/s} for sediment transport \\
\keyw{Q1} & \keyw{Discharge} & Discharge \unitbrackets{m\textsuperscript{3}/s} of the low flow simulation \\
\keyw{Q1} & \keyw{Reference} & Name of D-Flow FM map-file to be used for reference condition at Q1 \\
\keyw{Q1} & \keyw{WithMeasure} & Name of D-Flow FM map-file that includes the measure at Q1 \\
\keyw{Q2} & \keyw{Discharge} & Discharge \unitbrackets{m\textsuperscript{3}/s} of the transitional regime \\
\keyw{Q2} & \keyw{Reference} & Name of D-Flow FM map-file to be used for reference condition at Q2 \\
\keyw{Q2} & \keyw{WithMeasure} & Name of D-Flow FM map-file that includes the measure at Q2 \\
\keyw{Q3} & \keyw{Discharge} & Discharge \unitbrackets{m\textsuperscript{3}/s} of the high flow simulation \\
\keyw{Q3} & \keyw{Reference} & Name of D-Flow FM map-file to be used for reference condition at Q3 \\
\keyw{Q3} & \keyw{WithMeasure} & Name of D-Flow FM map-file that includes the measure at Q3 \\
\end{tabular}

The file names may be specified using relative or absolute paths.
The \keyw{Reference} and \keyw{WithMeasure} keywords are *not* used when the \keyw{Mode} equals \keyw{WAQUA export}; in that case the file name are standardized as \keyw{xyz\_<quantity>-zeta.00<1/2>.Q<i>}.

\subsubsection*{Example}

This example shows a complete analysis configuration file for a measure in the first branch/reach of the default \keyw{Dutch\_rivers.cfg} configuration.
It reports the default settings.
Only the \keyw{Version}, \keyw{Branch}, \keyw{Reach}, \keyw{Reference} and \keyw{WithMeasure} keywords are required for the full analysis.

\begin{Verbatim}
    [General]
      Version     = 1.0
      Mode        = D-Flow FM map
      Branch      = Bovenrijn & Waal
      Reach       = Bovenrijn                    km  859-867
      Qmin        = 1000.0
      Qbankfull   = 4000.0
      Ucrit       = 0.3
    
    [Q1]
      Discharge   = 1000.0
      Reference   = Measure42/Q1/Reference/DFM_OUTPUT_Q1/Q1_map.nc
      WithMeasure = Measure42/Q1/Updated/DFM_OUTPUT_Q1/Q1_map.nc
    
    [Q2]
      Discharge   = 4000.0
      Reference   = Measure42/Q2/Reference/DFM_OUTPUT_Q2/Q2_map.nc
      WithMeasure = Measure42/Q2/Updated/DFM_OUTPUT_Q2/Q2_map.nc
    
    [Q3]
      Discharge   = 6000.0
      Reference   = Measure42/Q3/Reference/DFM_OUTPUT_Q3/Q3_map.nc
      WithMeasure = Measure42/Q3/Updated/DFM_OUTPUT_Q3/Q3_map.nc
\end{Verbatim}

\section{simulation result files}

The WAQUA program writes the simulation results in a proprietary SDS file format.
The content of these result files can't be accessed directly from Python, so they need to be exported to a more accessible file format.
The previous WAQMORF program also required that the user extracted the cell centred velocity magnitude and water depth values from the the WAQUA SDS-output files write them to simple ASCII files with 6 columns specifying the x-coordinates, y-coordinates, the value of the quantity (labeled as z-coordinate), m-index, n-index, and cell number.
Only the "z" values and the m- and n-indices are used by the program.
The names of those files has been hardcoded, they should read \keyw{xyz\_<quantity>-zeta.00<1/2>.Q<i>} with the quantity name equal to 'velocity' or 'waterdepth', a 1 for the reference simulation and a 2 for the simulation with the measure implemented, and i the number of the discharge level.
\dfastmi supports these same files; an example of such file are given below.
In case of an anlysis based on D-Flow FM simulations, \dfastmi reads the results directly from the UGRID netCDF map-files.
These files may contain multiple time steps; the final time steps will be used for the analysis.
The mesh geometry is transferred from one of the simulation files to the \dfastmi results file.

\subsubsection*{Example}

\begin{Verbatim}
x,y,z,m,n,id
     197940.645,     431152.719,         0.0000,     2,     1,        11
     197874.004,     431430.516,         0.0000,     3,     1,        12
     197801.719,     431717.008,         0.0000,     4,     1,        13
     197727.047,     431985.375,         0.0000,     5,     1,        14
     197658.695,     432196.953,         0.0000,     6,     1,        15
     197610.234,     432323.859,         0.0000,     7,     1,        16
     197581.203,     432391.383,         0.0000,     8,     1,        17
     197559.625,     432439.234,         0.0000,     9,     1,        18
     197542.594,     432476.234,         0.0000,    10,     1,        19
     197528.668,     432506.070,         0.0000,    11,     1,        20
     197515.137,     432535.016,         0.0000,    12,     1,        21
     197505.176,     432556.258,         0.6156,    13,     1,        22
     197496.215,     432575.305,         0.8334,    14,     1,        23
     197485.773,     432597.516,         0.9794,    15,     1,        24
     197475.344,     432619.711,         1.0762,    16,     1,        25
     197465.578,     432640.484,         1.1305,    17,     1,        26
     197456.320,     432660.203,         1.1674,    18,     1,        27
     197447.242,     432679.563,         1.2085,    19,     1,        28
     197438.004,     432699.258,         1.2506,    20,     1,        29
     197428.273,     432720.000,         1.2913,    21,     1,        30
     197417.879,     432742.133,         1.3212,    22,     1,        31 

... continued ...
\end{Verbatim}


\section{report file}

\dfastmi will write a report of the analysis.
This report file is a simple text file consistent with the earlier reports written by WAQMORF.
The length and content of the report vary depending on the availability of the simulation result files and the language selected.


\section{spatial output file}

The WAQMORF program wrote the spatial output in SIMONA BOX-file format which could be visualize when combined with the curvilinear grid of the original simulation.
\dfastmi still generates such files when run based on WAQUA results.

\dfastmi generates one UGRID netCDF file containing the spatial results of the analysis.
The mesh information is copied from the D-Flow FM map file and the three data fields (erosion and sedimentation patterns for mean, minimum, and maximum impact) follow standardized conventions for data stored at cell centres (\keyw{face}-values) on an unstructured mesh.
As a result the data may be visualized using a number of visualization tools such as QUICKPLOT and QGIS.
