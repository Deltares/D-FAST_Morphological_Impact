\chapter{Steps in the analysis}\label{Chp:steps}

To apply \dfastmi one should carry out the following five steps.

\begin{enumerate}
\item Characterize the measure to be evaluated using \dfastmi.
Verify that it's appropriate to use \dfastmi instrument.
Determine the threshold discharge $Q_\text{thr}$ (at Lobith/Borgharen) at which the measure (indirectly) starts to influence the flow pattern in the main channel.

\item Determine the flow conditions for which \dflowfm simulations should be carried out based on the following table
\newline
\newline
\begin{tabular}{l|l|l}
river & location & flow conditions \\ \hline
Rhine branches & Lobith & 1300, 2000, 3000, 4000, 6000, 8000 \SI{}{\metre\cubed\per\second}\\
Meuse & Borgharen & 750, 1300, 1700, 2100, 2500, 3200 \SI{}{\metre\cubed\per\second}
\end{tabular}
\newline
\newline
It's not necessary to run simulations for conditions at which the measure doesn't influence the flow patterns (discharges below $Q_\text{thr}$).

\item Perform the necessary hydrodynamic simulations (note that for each condition two runs are needed: one reference run and one run with the measure)

\item Run \dfastmi to compute the characteristic bed level change per grid point in the main channel

\begin{itemize}
\item year-averaged mean value \unitbrackets{m}
\item maximum value \unitbrackets{m}
\item minimum value \unitbrackets{m}
\end{itemize}

\item{Visualize the characteristic bed level changes in a graph or on a map}
\end{enumerate}
