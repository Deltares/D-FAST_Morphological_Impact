\chapter{Introduction}

This document describes the code design considerations for the development of \dfastmi.
\dfastmi is the successor of WAQMORF.
The purpose of these tools is to provide a first estimate of the bed level changes in the main channel if local measures were to be implemented outside the main channel.
Such measure include the construction of side channels and other alterations of the floodplain and embankments.
The conceptual WAQMORF framework was originally developed by Sieben (2008).
To obtain the estimate of the bed level changes, the algorithm needs data on the river's behaviour.
These data include information on the probability distribution of discharge occurrences, the values of representative discharges, such as bankfull discharge, the general operation of any barriers, as well as the bedform celerities --- or propagation rates --- during low and high discharge conditions.
Based on these data the algorithm determines one, two or three characteristic discharges for which the user should provide the results of

\begin{itemize}
\item a hydrodynamic reference simulation and
\item a hydrodynamic simulation with the local measure implemented.
\end{itemize}

Based on these spatial data, the algorithm provides spatial estimates of the minimum, mean and maximum impact of the measure on the bed levels after one year.
The original WAQMORF tool performs the morphological analysis based on results exported from WAQUA simulations (hence the name WAQMORF).
The new \dfastmi tool performs the analysis based on D-Flow FM simulation results.