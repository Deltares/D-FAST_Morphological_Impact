\chapter{Developer Guide}

\section{Introduction}

This chapter gives an overview of how \dfastmi can be setup for development.

The development setup and configuration is done for usage of the visual studio code IDE. More information about visual studio code can be found on the internet.
The development of python code project within Deltares is mostly done with this tool. Visual Studio Code is free for private or commercial use. See the \href{https://code.visualstudio.com/license}{product license} for details.

\begin{Remark}
	\item This configuration can not be used to build a standalone executable with nuitka tooling. It is for development purposes only.
\end{Remark}

\section{Configuration}
The DFAST Morphological Impact developer environment can be configured with a batch script. This script will configure and install recommended tools. User input is required to download and install these tools. The batch file is located in the sub folder \textbf{BuildScripts} and is called \textbf{DevelopDfastmi.bat}.

The batch file will check for installed tooling for anaconda/miniconda. If not installed it will ask the user to install the anaconda/miniconda tooling for you. 

\begin{Remark}
	\item \textbf{PLEASE NOTE:} you need to close the command line and re-open / restart the batch script as it needs to initialize the conda tooling for the command line and powershell (used in VSCode).
\end{Remark}

The batch file will check for installed tooling for VSCode. If not installed it will ask the user to install the VSCode tooling for you. 

\begin{Remark}
	\item \textbf{PLEASE NOTE:} you need to close the command line and re-open / restart the batch script as it needs to initialize the VS Code tooling for the command line.
\end{Remark}


\section{Configuration files}
As stated earlier we use conda to create a clean python environment to develop the \dfastmi application in. We also use poetry to setup the needed packages for the development environment. Poetry uses a toml\footnote{TOML : Tom's Obvious Minimal Language. A config file format file} file for the configuration (\textbf{poetry.toml}) and a lock file (\textbf{poetry.lock}) which i have no clue why it is there. Also we use a \textbf{setup.py} file used by the setup tools package to setup the application for development. And last but not least a python tools / interpreter configuration \textbf{pyproject.toml}. This file configures the way python packages should behave.

\subsection{Conda environment}
We setup a conda environment on the client pc so we have the python interpreter we expect to be used by developers.
\begin{table}[]
	\caption{Default packages for conda environment with python interpreter v3.9.13.}
	\begin{tabular}{|l|l|}
		\hline
		\textbf{Package}    & \textbf{Remark} \\ \hline
		ca-certificates     &                 \\ \hline
		openssl             &                 \\ \hline
		pip                 &                 \\ \hline
		python              &                 \\ \hline
		setuptools          &                 \\ \hline
		sqlite              &                 \\ \hline
		tzdata              &                 \\ \hline
		vc                  &                 \\ \hline
		vs2015\_runtime     &                 \\ \hline
		wheel               &                 \\ \hline
	\end{tabular}
\end{table}

\subsubsection{Conda / Python packages needed for \dfastmi}
\begin{longtable}{|l|l|}
	\caption{Packages for developing.} \\
	\hline
	\textbf{Package} & \textbf{Remark} \\ \hline
	\endfirsthead
	%
	\multicolumn{2}{c}%
	{{\bfseries Packages for developing \thetable\ continued from previous page.}} \\
	\hline
	\textbf{Package} & \textbf{Remark} \\ \hline
	\endhead
	%
	click (8.1.7)                 &                 \\ \hline
	colorama (0.4.6)              &                 \\ \hline
	six (1.16.0)                  &                 \\ \hline
	attrs (23.2.0)                &                 \\ \hline
	certifi (2023.11.17)          &                 \\ \hline
	click-plugins (1.1.1)         &                 \\ \hline
	cligj (0.7.2)                 &                 \\ \hline
	exceptiongroup (1.2.0)        &                 \\ \hline
	iniconfig (2.0.0)             &                 \\ \hline
	munch (4.0.0)                 &                 \\ \hline
	numpy (1.26.3)                &                 \\ \hline
	packaging (23.2)              &                 \\ \hline
	pluggy (1.3.0)                &                 \\ \hline
	python-dateutil (2.8.2)       &                 \\ \hline
	pytz (2023.3.post1)           &                 \\ \hline
	tomli (2.0.1)                 &                 \\ \hline
	zipp (3.17.0)                 &                 \\ \hline
	cftime (1.3.0)                &                 \\ \hline
	contourpy (1.2.0)             &                 \\ \hline
	coverage (7.4.0)              &                 \\ \hline
	fiona (1.9.0)                 &                 \\ \hline
	fonttools (4.47.2)            &                 \\ \hline
	importlib-resources (6.1.1)   &                 \\ \hline
	cycler (0.12.1)               &                 \\ \hline
	intel-openmp (2021.4.0)       &                 \\ \hline
	kiwisolver (1.4.5)            &                 \\ \hline
	pandas (1.5.3)                &                 \\ \hline
	ordered-set (4.1.0)           &                 \\ \hline
	mypy-extensions (1.0.0)       &                 \\ \hline
	pathspec (0.12.1)             &                 \\ \hline
	pillow (10.2.0)               &                 \\ \hline
	platformdirs (4.1.0)          &                 \\ \hline
	pyparsing (3.1.1)             &                 \\ \hline
	pyproj (3.6.1)                &                 \\ \hline
	pyqt5-qt5 (5.15.2)            &                 \\ \hline
	pyqt5-sip (12.13.0)           &                 \\ \hline
	pytest (7.4.4)                &                 \\ \hline
	shapely (2.0.2)               &                 \\ \hline
	tbb (2021.7.1)                &                 \\ \hline
	typing-extensions (4.9.0)     &                 \\ \hline
	zstandard (0.22.0)            &                 \\ \hline
	black (22.12.0)               &                 \\ \hline
	geopandas (0.14.2)            &                 \\ \hline
	matplotlib (3.8.2)            &                 \\ \hline
	mkl (2021.4.0)                &                 \\ \hline
	nuitka (1.9.7)                &                 \\ \hline
	pyqt5 (5.15.10)               &                 \\ \hline
	netcdf4 (1.6.5)               &                 \\ \hline
	pytest-cov (4.1.0)            &                 \\ \hline
	teamcity-messages (1.32)      &                 \\ \hline
\end{longtable}

\section{Tooling}
DFAST Morphological Impact is using:
\begin{enumerate}
\item \keyw{anaconda} / \keyw{miniconda} /\keyw{conda} to setup its python environment. 
\item \keyw{poetry} to install the python packages for the environment
\item \keyw{VSCode} to edit python files, run the created (unit)tests, see/visualize the test coverage
\end{enumerate}

\subsection{Utility / install scripts}
We use several other batch scripts.
\begin{itemize}
	\item \keyw{CondaInstall.bat} used to install miniconda on the client pc, you need to restart the main script (DevelopDfastmi.bat) in a \textbf{new} command line prompt because the command line prompt environment is
	 updated after install.
	\item \keyw{VSCodeInstall.bat} used to install visual studio code on the client pc, you need to restart the main script (DevelopDfastmi.bat) in a \textbf{new} command line prompt because the command line prompt environment is updated after install.
\end{itemize}

\subsection{VSCode Extensions}
In VSCode we use extensions to find python unit test, visualize coverage and debug our code. To do this we install the following extensions.
\begin{longtable}{|l|l|}
	\caption{Extensions for developing with VS Code.} \\
	\hline
	\textbf{VS Code extension} & \textbf{Remark} \\ \hline
	\endfirsthead
	%
	\multicolumn{2}{c}%
	{{\bfseries Extensions for developing with VS Code \thetable\ continued from previous page}} \\
	\hline
	\textbf{VS Code extension} & \textbf{Remark} \\ \hline
	\endhead
	%
	Cameron.vscode-pytest                 		&                 \\ \hline
	donjayamanne.python-environment-manager     &                 \\ \hline
	hbenl.vscode-test-explorer                 	&                 \\ \hline
	littlefoxteam.vscode-python-test-adapter    &                 \\ \hline
	ms-python.pylint                 			&                 \\ \hline
	ms-python.python                 			&                 \\ \hline
	ms-python.vscode-pylance                 	&                 \\ \hline
	ms-vscode.test-adapter-converter            &                 \\ \hline
	ryanluker.vscode-coverage-gutters           &                 \\ \hline
\end{longtable}


\subsection{VSCode userfiles}
The user files can be found in subfolder \textbf{.vscode}
\begin{enumerate}
	\item \keyw{extensions.json} will reference the expected and recommended visual studio code extensions. It will not install them but inform the user what is missing to start developing.
	\item \keyw{launch.json} will configure VS Code what to do when an user will debug unit tests or start debugging the application.
	\item \keyw{settings.json} configured settings for visual studio code.
\end{enumerate}

\subsection{TOML}
This filetype is used by the python interpreter to configure the application. The TOML file itself has nothing to do with python, poetry or VS Code but it is a configuration file which is easy to read.
Added configuration settings to the \textbf{pyproject.toml} file
\begin{enumerate}
	\item In section [tool.coverage.run] \keyw{source = ["dfastmi"]} will inform VS code / python / python coverage tooling where to expect the coverage to apply to.
	\item In section [tool.pytest.ini\_options] \keyw{addopts = "---cov ---cov-report=term 
		\\ ---cov-report=xml:coverage-reports/coverage.xml"} will inform VS code / python / python test tooling to apply coverage and generate output report to.
\end{enumerate}


