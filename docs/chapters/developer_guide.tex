\chapter{Developer Guide}

\section{Introduction}

This chapter gives an overview of how \dfastmi can be setup for development. After following this guide the developer should be able to run the software, run unit tests, generate the code coverage and debug through the tests and software using tools and development programs.

The development setup and configuration is done for usage of the visual studio code IDE. More information about \href{https://code.visualstudio.com/}{visual studio code} can be found on the internet.
The development of python code project within Deltares is mostly done with this tool. Visual Studio Code is free for private or commercial use. See the \href{https://code.visualstudio.com/license}{product license} for details.

\begin{Remark}
	\item This configuration can not be used to build a standalone executable with nuitka tooling. It is for development purposes only.
\end{Remark}

\section{Configuration}
The DFAST Morphological Impact developer environment can be configured with a batch script. This script will configure and install recommended tools. User input is required to download and install these tools. The batch file is located in the sub folder \textbf{BuildScripts} and is called \textbf{DevelopDfastmi.bat}.

The batch file will check for installed tooling for anaconda/miniconda. If not installed it will ask the user to install the anaconda/miniconda tooling for you. 

\begin{Remark}
	\item \textbf{PLEASE NOTE:} you need to close the command line and re-open / restart the batch script as it needs to initialize the conda tooling for the command line and powershell (used in VSCode).
\end{Remark}

The batch file will check for installed tooling for VSCode. If not installed it will ask the user to install the VSCode tooling for you. 

\begin{Remark}
	\item \textbf{PLEASE NOTE:} you need to close the command line and re-open / restart the batch script as it needs to initialize the VS Code tooling for the command line.
\end{Remark}

After checking (and installing if needed) the VSCode code development application the batch script will install several important VSCode extensions needed for quick development.


\section{Configuration files}
As stated earlier we use conda to create a clean python environment to develop the \dfastmi application in. We also use poetry to setup the needed packages for the development environment. Poetry uses a toml\footnote{TOML : Tom's Obvious Minimal Language. A config file format file} file for the configuration (\textbf{poetry.toml}) and a lock file (\textbf{poetry.lock}) which means we ran the install command of poetry before with these settings. More information about the poetry.lock can be found here: \href{https://python-poetry.org/docs/basic-usage/#installing-without-poetrylock}. Also we use a \textbf{setup.py} file used by the setup tools package to setup the application for development. And last but not least a python tools / interpreter configuration \textbf{pyproject.toml}. This file configures the way python packages should behave.

\subsection{Conda environment}
We setup a conda environment so we have the python interpreter we expect to be used by developers.
We do this so the developer always can create isolated environments that contain a specific version of Python and all the required dependencies for DFast MI. This prevents conflicts between packages and ensures that DFast MI development environment is reproducible.
\begin{table}[]
	\caption{Default installed packages for conda environment with python interpreter v3.9.13, installed by conda.}
	\begin{tabular}{|l|l|}
		\hline
		\textbf{Package}    & \textbf{Remark} \\ \hline
		ca-certificates     & Certificates for use with other packages.\\ \hline
		openssl             & OpenSSL is an open-source implementation of the SSL and TLS protocols. \\ \hline
		pip                 & PyPA recommended tool for installing Python packages. \\ \hline
		python              & General purpose programming language. V3.9.13 \\ \hline
		setuptools          & Download, build, install, upgrade, and uninstall Python packages. \\ \hline
		sqlite              & Implements a self-contained, zero-configuration, SQL database engine \\ \hline
		tzdata              & The Time Zone Database (called tz, tzdb or zoneinfo) data. \\ \hline
		vc                  & A meta-package to impose mutual exclusivity among software built with different
		\\ & VS versions. \\ \hline
		vs2015\_runtime     & A backwards compatible meta-package. See vc14\_runtime for the new package. \\ \hline
		wheel               & A built-package format for Python. \\ \hline
	\end{tabular}
\end{table}

\subsubsection{Conda / Python packages needed for \dfastmi}
\begin{longtable}{|l|l|}
	\caption{Packages for developing.} \\
	\hline
	\textbf{Package} & \textbf{Remark} \\ \hline
	\endfirsthead
	%
	\multicolumn{2}{c}%
	{{\bfseries Packages for developing \thetable\ continued from previous page.}} \\
	\hline
	\textbf{Package} & \textbf{Remark} \\ \hline
	\endhead
	%
	click (8.1.7)                 &                 \\ \hline
	colorama (0.4.6)              &                 \\ \hline
	six (1.16.0)                  &                 \\ \hline
	attrs (23.2.0)                &                 \\ \hline
	certifi (2023.11.17)          &                 \\ \hline
	click-plugins (1.1.1)         &                 \\ \hline
	cligj (0.7.2)                 &                 \\ \hline
	exceptiongroup (1.2.0)        &                 \\ \hline
	iniconfig (2.0.0)             &                 \\ \hline
	munch (4.0.0)                 &                 \\ \hline
	numpy (1.26.3)                & NumPy is the fundamental package for scientific computing
	\\ & with Python.\\ \hline
	packaging (23.2)              &                 \\ \hline
	pluggy (1.3.0)                &                 \\ \hline
	python-dateutil (2.8.2)       &                 \\ \hline
	pytz (2023.3.post1)           &                 \\ \hline
	tomli (2.0.1)                 &                 \\ \hline
	zipp (3.17.0)                 &                 \\ \hline
	cftime (1.3.0)                &                 \\ \hline
	contourpy (1.2.0)             &                 \\ \hline
	coverage (7.4.0)              & Code coverage measurement for Python. \\ \hline
	fiona (1.9.0)                 & Fiona streams simple feature data to and from GIS formats like 
	\\ & GeoPackage and Shapefile.
	\\ & Fiona can read and write real-world data using 
	\\ & multi-layered GIS formats, zipped and in-memory 
	\\ & virtual file systems, from files on your hard 
	\\ & drive( or in cloud storage).	
	\\ & Fiona depends on GDAL but is different from GDAL’s
	\\ &  own bindings. Fiona is designed to be highly 
	\\ & productive and to make it easy to write code 
	\\ & which is easy to read. \\ \hline
	fonttools (4.47.2)            &                 \\ \hline
	importlib-resources (6.1.1)   &                 \\ \hline
	cycler (0.12.1)               &                 \\ \hline
	intel-openmp (2021.4.0)       &                 \\ \hline
	kiwisolver (1.4.5)            &                 \\ \hline
	pandas (1.5.3)                & Python data analysis toolkit. \\ \hline
	ordered-set (4.1.0)           &                 \\ \hline
	mypy-extensions (1.0.0)       &                 \\ \hline
	pathspec (0.12.1)             &                 \\ \hline
	pillow (10.2.0)               &                 \\ \hline
	platformdirs (4.1.0)          &                 \\ \hline
	pyparsing (3.1.1)             &                 \\ \hline
	pyproj (3.6.1)                &                 \\ \hline
	pyqt5-qt5 (5.15.2)            & This package contains the subset of a Qt installation that is
	\\ & required by PyQt5. \\ \hline
	pyqt5-sip (12.13.0)           &                 \\ \hline
	pytest (7.4.4)                & The pytest framework makes it easy to write small tests,
	\\ & yet scales to support complex functional testing
	\\ & for applications and libraries. \\ \hline
	shapely (2.0.2)               & Shapely is a package for manipulation and analysis of planar 
	\\ & geometric objects. \\ \hline
	tbb (2021.7.1)                &                 \\ \hline
	typing-extensions (4.9.0)     &                 \\ \hline
	zstandard (0.22.0)            &                 \\ \hline
	black (22.12.0)               & A Python code formatter. \\ \hline
	geopandas (0.14.2)            & GeoPandas is a project to add support for geographic data to 
	\\ & pandas objects.
	\\ & 
	\\ & The goal of GeoPandas is to make working with geospatial data 
	\\ & in python easier. It combines the capabilities of pandas and shapely, 
	\\ & providing geospatial operations in pandas and a high-level interface 
	\\ & to multiple geometries to shapely. GeoPandas enables you to easily 
	\\ & do operations in python that would otherwise require a spatial 
	\\ & database such as PostGIS.                \\ \hline
	matplotlib (3.8.2)            &                 \\ \hline
	mkl (2021.4.0)                & Intel® oneAPI Math Kernel Library (Intel® oneMKL) is a computing 
	\\ & math library. \\ \hline
	nuitka (1.9.7)                & Nuitka is the Python compiler. It is written in Python. It is a seamless
	\\ & replacement or extension to the Python interpreter and compiles 
	\\ & every construct that CPython when itself run with that Python version.
	\\ &
	\\ & It then executes uncompiled code and compiled code together in an
	\\ & extremely compatible manner.
	\\ &
	\\ & Nuitka translates the Python modules of \dfastmi into
	\\ & a C level program that then uses libpython and static C files  
	\\ & of its own to execute in the same way as CPython does. \\ \hline
	pyqt5 (5.15.10)               & Qt is set of cross-platform C++ libraries that implement high-level APIs 
	\\ & for accessing many aspects of modern desktop. 
	\\ & These include traditional UI development.
	\\ &
	\\ & PyQt5 is a comprehensive set of Python bindings for Qt v5. 
	\\ & It is implemented as more than 35 extension modules and enables 
	\\ & Python to be used as an alternative application development language \\ \hline
	netcdf4 (1.6.5)               & This module can read and write netCDF files. \\ \hline
	pytest-cov (4.1.0)            & This plugin produces coverage reports. \\ \hline
	teamcity-messages (1.32)      & \\ \hline
\end{longtable}

\section{Tooling}
DFAST Morphological Impact is using:
\begin{enumerate}
\item \keyw{anaconda} / \keyw{miniconda} /\keyw{conda} to setup its python environment. 
\item \keyw{poetry} to install the python packages for the environment
\item \keyw{VSCode} to edit python files, run the created (unit)tests, see/visualize the test coverage
\end{enumerate}

\subsection{Utility / install scripts}
We use several other batch scripts.
\begin{itemize}
	\item \keyw{CondaInstall.bat} used to install miniconda, you need to restart the main script (DevelopDfastmi.bat) in a \textbf{new} command line prompt because the command line prompt environment is
	 updated after install.
	\item \keyw{VSCodeInstall.bat} used to install visual studio code on the client pc, you need to restart the main script (DevelopDfastmi.bat) in a \textbf{new} command line prompt because the command line prompt environment is updated after install.
\end{itemize}

\subsection{VSCode Extensions}
In VSCode we use extensions to find python unit test, visualize coverage and debug our code. To do this we install the following extensions.
\begin{longtable}{|l|l|}
	\caption{Extensions for developing with VS Code.} \\
	\hline
	\textbf{VS Code extension} & \textbf{Remark} \\ \hline
	\endfirsthead
	%
	\multicolumn{2}{c}%
	{{\bfseries Extensions for developing with VS Code \thetable\ continued from previous page}} \\
	\hline
	\textbf{VS Code extension} & \textbf{Remark} \\ \hline
	\endhead
	%
	Cameron.vscode-pytest                 		& VS Code Pytest Extension. \\ \hline
	donjayamanne.  & A Visual Studio Code extension that provides the ability \\
	python-environment-manager & to via and manage all of your Python environments \& 
	\\ & packages from a single place. \\ \hline
	ms-python.pylint                 			& A Visual Studio Code extension with support for the Pylint linter. 
	\\ & For more information on Pylint, see \href{https://pylint.readthedocs.io/}{pylint documentation}. \\ \hline
	ms-python.python                 			& A Visual Studio Code extension with rich support for the
	\\ & Python language including features such 
	\\ & as code navigation, code formatting, refactoring, variable explorer. \\ \hline
	ms-python.vscode-pylance                 	& To provide performant Python language support. \\ \hline
	ryanluker.vscode-coverage-gutters           & Display test coverage generated by lcov or xml in your editor 
	\\ & over your developed code. \\ \hline
\end{longtable}


\subsection{VSCode userfiles}
The user files can be found in subfolder \textbf{.vscode}
\begin{enumerate}
	\item \keyw{extensions.json} will reference the expected and recommended visual studio code extensions. It will not install them but inform the user what is missing to start developing.
	\item \keyw{launch.json} will configure VS Code what to do when an user will debug unit tests or start debugging the application.
	\item \keyw{settings.json} configured settings for visual studio code.
\end{enumerate}

\subsection{TOML}
This filetype is used by the python interpreter to configure the application. The TOML file itself has nothing to do with python, poetry or VS Code but it is a configuration file which is easy to read. For more information see \href{https://coverage.readthedocs.io/en/latest/config.html#toml-syntax}.
Added configuration settings to the \textbf{pyproject.toml} file
\begin{enumerate}
	\item In section [tool.coverage.run] \keyw{source = ["dfastmi"]} will inform VS code / python / python coverage tooling where to expect the coverage to apply to.
	\item In section [tool.pytest.ini\_options] \keyw{addopts = "---cov ---cov-report=term 
		\\ ---cov-report=xml:coverage-reports/coverage.xml"} will inform VS code / python / python test tooling to apply coverage and generate output report to.
\end{enumerate}


