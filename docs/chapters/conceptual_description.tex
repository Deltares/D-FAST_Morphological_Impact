\chapter{Conceptual framework}

This chapter describes the conceptual framework of the method used by \dfastmi and the derivation of the river parameters used for the Dutch rivers.

\section{Characteristics of bed level changes due to river measures}

A river measure, i.e.~a local river adjustment (typically a flow capacity enlarging adjustment) outside the main channel, may result in changes in the flow pattern within the main channel.
The bed levels within the main channel will react to such changes depending on the

\hspace{1cm}\emph{magnitude, duration and length scale of these flow pattern changes in the main channel.}

Hence, the core focus of the rule-of-thumb is to determine and characterize the changes in the flow pattern within the main channel.
Here, we define the main channel as the part of the river between the groyne heads.

At the upstream end of the floodplain adjustment part of the main channel discharge will leave the main channel, which will flow through the enlarged floodplain, secondary channel or bank area, and return to the main channel at the downstream end of the project area.
The resulting changes in the main channel flow pattern cause sedimentation in the main channel at the upstream "offtake" of the new outflow and erosion at the downstream "confluence" when the flow enters the main channel again.
Because the river discharge varies, the pattern will also vary during the year
As a result a part of the main channel bed level adjustment is stationary and a part of the adjustment is transitional and it will migrate downstream.
A reduction in the water depth may thus not only be observed locally where the flow diverges into the flood plain, but also downstream thereof.

The bed level change due to river enlargement can largely be characterized as follows

\begin{itemize}
\item The magnitude of the bed level change varies over the seasons; the maximum change is observed at the end of the flood season and the adjustment reaches a minimum at the end of the low flow period.

\item Local evaluation of the bed level changes is insufficient because of partial downstream migration of the bed level changes.

\item The maximum impact is observed on that side of the river which is closest to the adjustment.
\end{itemize}

Because part of the bed level changes migrate downstream, the morphological impact on projects located downstream may be increased.
After all, the maximum bed level change of one single river enlarging measure is reached at the end of the flood period and it's the result of that flood period.
However, when a series of measures along the river are combined (for instance, lowering of the groynes along a reach) the downstream bed level changes may be increased by accumulation of downstream migrating bed waves, such that the accumulated bed level change may be significantly larger.
In other words, river measures extending over longer reaches may result in a larger morphodynamic response than comparable but more isolated measures.

The aforementioned observations result in the following premisses.
The rule-of-thumb described in the following sections

\begin{itemize}
\item will only be valid for the estimation of the impact of local measures with a length of at most one flood plain,

\item should take into account seasonal variations, and

\item needs to be estimate the spatial pattern of the bed level change.
\end{itemize}


\section{Characterization of the changes in the main channel flow pattern}

When a measure in the floodplains does not include something like a permanently active secondary channel, the flow pattern in the main channel will only be affected during flood conditions.
If on the other hand, such a secondary channel (or other measure active under medium to low flow conditions such as groyne adjustments) is included in the design then the flow pattern will also be influenced during transitional or even low flow conditions.
It's therefore critical to determine the effect of the measure on the main channel flow pattern at different discharge levels.

Secondary channels may have a major impact on the morphodynamics of the main channel.
The magnitude of the discharge through the secondary channel varies as function of the river discharge.
Because the hydraulic conditions of the secondary channel can change over time (e.g.~due to vegetation, siltation, or erosion) the discharge through a secondary channel is typically controlled by some hydraulic structure.
The typical characteristics of a weir (overlaat, drempel) and orifice (onderlaat, duiker) are shown in \autoref{Fig1}.

\begin{figure}
\includegraphics[width=\columnwidth]{figures/Fig1.png}
\caption{Schematic of the controlled discharge through a secondary channel.}
\label{Fig1}
\end{figure}

The discharge through a secondary channel is characterized by a number of special conditions as illustrated by \autoref{Fig1}:

\begin{enumerate}
\item the beginning of flow through the secondary channel
\item the bankfull secondary channel
\item the fully developed flow in the secondary channel and surrounding floodplain.
\end{enumerate}

When the discharge through the secondary channel is controlled by an orifice, the fully flooded condition may add an additional characteristic discharge.

In order to characterize the flow patterns during flood it's necessary to distinguish the condition in which the flood plains just start to carry flows, and the condition of fully developed flow in the flood plains.
Based on results of 1D simulations shown in \autoref{Fig2a} it has been concluded that for the Rhine branches a discharge of 6000 m\textsuperscript{3}/s at Lobith can be used as the boundary between those two conditions.
The flow through the bank zone (\autoref{Fig2b}) is also largely developed for discharges at Lobith larger than
6000 m\textsuperscript{3}/s.

\begin{figure}
\includegraphics[width=\columnwidth]{figures/Fig2a.png}
\caption{Reach-averaged discharge fraction through the main channel (based on 1D model of the Rhine branches).}
\label{Fig2a}
\end{figure}

\begin{figure}
\includegraphics[width=\columnwidth]{figures/Fig2b.png}
\caption{Reach-averaged discharge fraction through the bank zone (based on 1D model of the Rhine branches).}
\label{Fig2b}
\end{figure}

These characteristic discharge values can be used to schematize the river hydrograph effectively by means of a limited number of discrete discharge conditions.
Three discharge blocks turn out to be sufficient for many measures as may bed concluded from \autoref{Tab1}.
That is why it has been concluded that \emph{three flow conditions are sufficient to schematize the yearly hydrograph for the purpose of estimating the morphological impact of local measures}.

\begin{table}
\small
%\includegraphics[width=\columnwidth]{figures/Tab1a.png}
\begin{tabular}{p{\columnwidth/4-12pt}|p{\columnwidth/4-12pt}|p{\columnwidth/4-12pt}|p{\columnwidth/4-12pt}}
 & first block & second block & third block \\ \hline
periodically flowing secondary channel with weir & from minimum river flow to minimum discharge for secondary channel & from minimum discharge for secondary channel to bankfull secondary channel & from bankfull secondary channel to fully developed flood plain flow \\ \hline
periodically flowing secondary channel with orifice & from minimum river flow to minimum discharge for secondary channel & from minimum discharge for secondary channel to fully flooded orifice & from fully flooded orifice to fully developed flood plain flow \\ \hline
permanently flowing secondary channel & from minimum river flow to bankfull secondary channel & from bankfull secondary channel to 6000 m\textsuperscript{3}/s at Lobith & from 6000 m\textsuperscript{3}/s at Lobith to fully developed flood plain flow \\ \hline
local lowering of groynes (less than flood plain length) & from minimum river flow to flooded groynes & from flooded groynes to 6000 m\textsuperscript{3}/s at Lobith & from 6000 m\textsuperscript{3}/s at Lobith to fully developed flood plain flow \\ \hline
lowering of the flood plain & from minimum river flow to minimum discharge for new flood plain threshold & from new flood plain threshold to 6000 m\textsuperscript{3}/s at Lobith & from 6000 m\textsuperscript{3}/s at Lobith to fully developed flood plain flow \\ \hline
lowering of levees & from minimum river flow to minimum discharge for new levee threshold & from new levee threshold to 6000 m\textsuperscript{3}/s at Lobith & from 6000 m\textsuperscript{3}/s at Lobith to fully developed flood plain flow \\
\end{tabular}

\caption{Overview of discharge conditions for various measures in a free flowing Rhine branch.}
\label{Tab1}
\end{table}

For a river reach in which water levels (and hence flow conditions) are controlled by means of barriers for a period $T_\text{stuw}$, the minimum river discharge is determined by the river discharge at which the barriers are opened.
\autoref{Fig6} shows that the barriers in the Nederrijn are opened at a discharge of 1500 m\textsuperscript{3}/s in the Bovenrijn at Lobith (this value is exceeded during 57 \% of the year) and that the branch is approximately free flowing for discharges above 2200 m\textsuperscript{3}/s in the Bovenrijn at Lobith (exceeded during 33 \% of the year).

\begin{table}
%\includegraphics[width=\columnwidth]{figures/Tab1b.png}
\begin{tabular}{p{\columnwidth/4-12pt}|p{\columnwidth/4-12pt}|p{\columnwidth/4-12pt}|p{\columnwidth/4-12pt}}
 & first block & second block & third block \\ \hline
periodically flowing secondary channel with weir & from 1500 m\textsuperscript{3}/s at Lobith to minimum discharge for secondary channel & from minimum discharge for secondary channel to bankfull secondary channel & from bankfull secondary channel to fully developed flood plain flow \\ \hline
periodically flowing secondary channel with orifice & from 1500 m\textsuperscript{3}/s at Lobith to minimum discharge for secondary channel & from minimum discharge for secondary channel to fully flooded orifice & from fully flooded orifice to fully developed flood plain flow \\ \hline
permanently flowing secondary channel & from 1500 m\textsuperscript{3}/s at Lobith to bankfull secondary channel (about 4000 m\textsuperscript{3}/s at Lobith) & from bankfull secondary channel (about 4000 m\textsuperscript{3}/s at Lobith) to 6000 m\textsuperscript{3}/s at Lobith & from 6000 m\textsuperscript{3}/s at Lobith to fully developed flood plain flow \\ \hline
local lowering of groynes (less than flood plain length) & from 1500 m\textsuperscript{3}/s at Lobith to flooded groynes & from flooded groynes to 6000 m\textsuperscript{3}/s at Lobith & from 6000 m\textsuperscript{3}/s at Lobith to fully developed flood plain flow \\ \hline
lowering of the flood plain & from 1500 m\textsuperscript{3}/s at Lobith to minimum discharge for new flood plain threshold & from new flood plain threshold to 6000 m\textsuperscript{3}/s at Lobith & from 6000 m\textsuperscript{3}/s at Lobith to fully developed flood plain flow \\ \hline
lowering of levees & from 1500 m\textsuperscript{3}/s at Lobith to minimum discharge for new levee threshold & from new levee threshold to 6000 m\textsuperscript{3}/s at Lobith & from 6000 m\textsuperscript{3}/s at Lobith to fully developed flood plain flow \\
\end{tabular}

\caption{Overview of discharge conditions for various measures along the Nederrijn.}
\label{Tab2}
\end{table}

\begin{table}
%\includegraphics[width=\columnwidth]{figures/Tab1c.png}
\begin{tabular}{p{\columnwidth/4-12pt}|p{\columnwidth/4-12pt}|p{\columnwidth/4-12pt}|p{\columnwidth/4-12pt}}
 & first block & second block & third block \\ \hline
periodically flowing secondary channel with weir & from 1000 m\textsuperscript{3}/s at Borgharen to minimum discharge for secondary channel & from minimum discharge for secondary channel to bankfull secondary channel & from bankfull secondary channel to fully developed flood plain flow \\ \hline
periodically flowing secondary channel with orifice & from 1000 m\textsuperscript{3}/s at Borgharen to minimum discharge for secondary channel & from minimum discharge for secondary channel to fully flooded orifice & from fully flooded orifice to fully developed flood plain flow \\ \hline
permanently flowing secondary channel & from 1000 m\textsuperscript{3}/s at Borgharen to bankfull secondary channel & from bankfull secondary channel to 2000 m\textsuperscript{3}/s at Borgharen & from 2000 m\textsuperscript{3}/s at Borgharen to fully developed flood plain flow \\ \hline
local lowering of groynes (less than flood plain length) & from 1000 m\textsuperscript{3}/s at Borgharen to flooded groynes & from flooded groynes to 2000 m\textsuperscript{3}/s at Borgharen & from 2000 m\textsuperscript{3}/s at Borgharen to fully developed flood plain flow \\ \hline
lowering of the flood plain & from 1000 m\textsuperscript{3}/s at Borgharen to minimum discharge for new flood plain threshold & from new flood plain threshold to 2000 m\textsuperscript{3}/s at Borgharen & from 2000 m\textsuperscript{3}/s at Borgharen to fully developed flood plain flow \\ \hline
lowering of levees & from 1000 m\textsuperscript{3}/s at Borgharen to minimum discharge for new levee threshold & from new levee threshold to 2000 m\textsuperscript{3}/s at Borgharen & from 2000 m\textsuperscript{3}/s at Borgharen to fully developed flood plain flow \\
\end{tabular}

\caption{Overview of discharge conditions for various measures along the Meuse.}
\label{Tab3}
\end{table}

\begin{figure}
\includegraphics[width=\columnwidth]{figures/Fig3a.png}
\caption{Discharge exceedance curve for the Bovenrijn at Lobith (1970-2000).}
\label{Fig3a}
\end{figure}

\begin{figure}
\includegraphics[width=\columnwidth]{figures/Fig3b.png}
\caption{Discharge exceedance curve for the Meuse at Borgharen (1975-2007).}
\label{Fig3b}
\end{figure}

In order to quickly estimate the yearly exceedance period for a given discharge, the discharge exceedance curve for the Bovenrijn has been approximated by a curve fitted through the observational data for the period 1970-2000 (\autoref{Fig3a}).
The number of days that the discharge $Q_\text{Bovenrijn}$ in the Bovenrijn at lobith is exceeded is given by

\begin{equation}
\label{Eq1a}
T_\text{exceedance, Bovenrijn} = 365 e^{\left ( \frac{800 -  Q_\text{Bovenrijn}}{1280} \right )}
\end{equation}

For the Meuse, in which --- due to imbrication (Grensmaas) and barriers (Zandmaas) --- the higher discharges are more important for the morphological development, the exceedance period for discharges at Borgharen can be estimated as

\begin{equation}
\label{Eq1b}
T_\text{exceedance, Meuse} = 365 e^{\left ( \frac{-  Q_\text{Borgharen}}{300} \right )}
\end{equation}

\section{Scientific background of the relaxation model for morphological change}

The rule-of-thumb used to determine the first estimate of the morphological effects within the main channel is based on a highly simplified model of the morphodynamics.
That model and the resulting rule-of-thumb are described in this section.

We start by assuming a quasi-stationary flow pattern and an outflow of discharge and sediment from the main channel while the local water level is independent of the hydraulic and morphological changes in the main channel (a \emph{rigid-lid} approximation).

\begin{figure}
\includegraphics[width=\columnwidth]{figures/Fig4.png}
\caption{Schematic representation of the outflow of water and sediment}
\label{Fig4}
\end{figure}

The response of the main channel is represented here by a bed level change $\Delta z_b$ \unitbrackets{m} (raise or lowering) which is small relative to the local water depth.
The mass balance of the water and the sediment in the main channel can be written as
%
\begin{align}
&\text{water} & \pdiff{q_\text{mc}}{s} = -q_\text{out} \label{Eq2} \\
&\text{sediment} & \pdiff{z_b}{t} + \pdiff{s_\text{mc}}{s} = -s_\text{out} \label{Eq3}
\end{align}
%
in which
%
\begin{symbollist}
\item[$q_\text{mc}$] main channel unit discharge \unitbrackets{m\textsuperscript{2}/s}
\item[$q_\text{out}$] \emph{outflow of} unit discharge \emph{to} the flood plain per unit length \unitbrackets{m\textsuperscript{2}/sm}
\item[$s$] streamwise coordinate \unitbrackets{m}
\item[$z_b$] bed level change due to measure \unitbrackets{m}
\item[$s_\text{mc}$] main channel unit sediment transport including pores \unitbrackets{m\textsuperscript{2}/s}
\item[$s_\text{out}$] \emph{outflow of} unit sediment transport including pores \emph{to} the flood plain per unit length \unitbrackets{m\textsuperscript{2}/sm}
\end{symbollist}

Note that the outflow of sediment is assumed to be a consequence of the outflow of discharge only; the conceptual framework does not allow for selective removal of only sediments.	
The unit sediment transport capacity in the main channel, $s_\text{mc}$, is approximated by $s_\text{mc} = m \left ( q_\text{mc} / h \right )^b$ with $h$ the water depth \unitbrackets{m} in the main channel, $m$ a dimensionless calibration factor and a dimensionless sediment transport exponent $b$ about 5 \citep{Engelundh67}.
Using this approximation the gradients in the sediment transport capacity can be written as
%
\begin{align}
\pdiff{s_\text{mc}}{s} &= m b \left ( \frac{q_\text{mc}}{h} \right )^{b-1} \pdiff{(q_\text{mc} / h)}{s} \\
&= m b \left ( \frac{q_\text{mc}}{h} \right )^{b-1} \left [ \frac{1}{h} \pdiff{q_\text{mc}}{s} - \frac{q_\text{mc}}{h^2} \pdiff{h}{s} \right ] \\
&= b \frac{s_\text{mc}}{q_\text{mc}} \pdiff{q_\text{mc}}{s} - b \frac{s_\text{mc}}{h} \pdiff{h}{s} \\
&\approx b \frac{s_\text{mc}}{q_\text{mc}} \pdiff{q_\text{mc}}{s} + b \frac{s_\text{mc}}{h} \pdiff{z_b}{s}
\label{Eq4}
\end{align}
%
if we assume that the water level gradient is negligible ($\pdiff{z_w}{s} \approx 0$) such that $\pdiff{h}{s} \approx -\pdiff{z_b}{s}$.
Substitution of \autoref{Eq4} into \autoref{Eq3} gives
%
\begin{equation}
\pdiff{z_b}{t} + b \frac{s_\text{mc}}{q_\text{mc}} \pdiff{q_\text{mc}}{s} + b \frac{s_\text{mc}}{h} \pdiff{z_b}{s} = -s_\text{out}
\label{Eq5a}
\end{equation}
%
which after substitution of \autoref{Eq2} in the second term becomes
%
\begin{equation}
\frac{h}{b s_\text{mc}} \pdiff{z_b}{t} + \pdiff{z_b}{s} = h \frac{q_\text{out}}{q_\text{mc}} - h \frac{s_\text{out}}{b s_\text{mc}}
\label{Eq5}
\end{equation}

The dynamic bed level changes will develop starting from the upstream end of the measure.
The length $L$ over which this sedimentation occurs (i.e.~varies from 0 upstream to maximum amount $\Delta z_b$ downstream) corresponds to the reach over which discharge leaves the main channel and in which the flow pattern is adjusted by the reduced discharge as sketched in \autoref{Fig4}.
Hence, this length $L$ may thus be significantly shorter than the length of the measure or the distance between the inflow and outflow openings of a secondary channel.
The integration of \autoref{Eq5} over this length $L$ results in a relaxation model for the dynamic bed level change due to the measure
%
\begin{equation}
\frac{h}{b s_\text{mc}} \int_0^L \pdiff{z_b}{t} ds + \int_0^L \pdiff{z_b}{s} ds = \int_0^L \left [ h \frac{q_\text{out}}{q_\text{mc}} - h \frac{s_\text{out}}{b s_\text{mc}} \right ] ds
\label{Eq5a}
\end{equation}
%
\begin{equation}
\frac{h}{b s_\text{mc}} L^* \pdiff{\Delta z_b}{t} + \Delta z_b = - h \frac{\Delta q_\text{mc}}{q_\text{mc}} + h \frac{\Delta s_\text{mc}}{b s_\text{mc}}
\label{Eq5b}
\end{equation}
%
which $\Delta q_\text{mc}$ is the total change in unit discharge in the main channel due to the measure \unitbrackets{m\textsuperscript{2}/s} and $\Delta s_\text{mc}$ is the total change in unit sediment transport due to the measure \unitbrackets{m\textsuperscript{2}/s}.
Please note that $q_\text{out}$ was defined above positive for fluxes towards the flood plain, however, the resulting change $\Delta q_\text{mc}$ in the unit discharge within the main channel will be negative for such conditions (equivalently for $s_\text{out}$ and $\Delta s_\text{mc}$); this introduces a sign change on the right hand side.
%
\begin{equation}
\diff{\Delta z_b}{t} = \frac{\Delta z_{b,\text{eq}} - \Delta z_b}{T_m}
\label{Eq5dif}
\end{equation}
%
with a morphological time scale \unitbrackets{s}
%
\begin{equation}
T_m \approx \frac{h L}{b s_\text{mc}}
\label{Eq5T}
\end{equation}
%
and an equilibrium bed level change \unitbrackets{m} given by
%
\begin{equation}
\Delta z_{b,\text{eq}} = -h \left ( \frac{\Delta q_\text{mc}}{q_\text{mc}} - \frac{\Delta s_\text{mc}}{b s_\text{mc}} \right )
\label{Eq6}
\end{equation}
%
This relaxation behaviour can also be observed in the results of the numerical models (both 1D and 2D).

The rule-of-thumb is derived by posing that the bed level change $\Delta z_b$ can be interpreted as the bed level change due to a measure over a distance $L$ along a stream line starting from the upstream end of the measure.
The equilibrium value $\Delta z_{b,\text{eq}}$ depends according \autoref{Eq6} on the (original) water depth, the relative change in unit discharge and the relative change in the sediment transport.
The latter term is the result of the sediment flux into the secondary channel which dampens the effect of reduced sediment transport capacity in the main channel.
Ignoring this relatively minor dampening term results in a more conservative estimate.
For measures over a short distance (i.e.~for the type of measures for which the rule-of-thumb is applicable) the water level changes are an order of magnitude smaller than the bed level changes.
Therefore, one can rewrite \autoref{Eq6} to state that the equilibrium bed level change $\Delta z_{b,\text{eq}}$ equals the product of the water depth and the relative change in the main channel velocity $u$ \unitbrackets{m/s}:
%
\begin{equation}
\Delta z_{b,\text{eq}} \approx -h \left ( \frac{\Delta u}{u} \right )
\label{Eq6v2}
\end{equation}
%
A similar result can be obtained for graded bed material (sand/gravel mixtures) if it's assumed that the individual sediment fractions don't influence each other's mobility (Appendix C in \citet{Waterdienst2008}).
Such an assumption is valid at sufficiently high bed shear stresses, such as during flood conditions.

\section{The relaxation model applied to seasonal variability}

The bed level development during a relaxation period is (as a solution of \autoref{Eq5dif}) given by
%
\begin{equation}
z_{b,i} = z_{b,i} (0) + [z_{b,i,\text{eq}} - z_{b,i}(0)](1 - e^{-t/T_{m,i}})
\label{Eq7}
\end{equation}
%
with
%
\begin{symbollist}
\item[$z_{b,i}$] \unitbrackets{m} the morphological effect of the measure during period $i$
\item[$z_{b,i}(0)$] \unitbrackets{m} the morphological effect at the start of period $i$
\item[$z_{b,i,\text{eq}}$] \unitbrackets{m} the equilibrium effect of the measure during period $i$
\item[$t$] \unitbrackets{day} time
\item[$T_{m,i}$] \unitbrackets{day} the morphological time scale during period $i$
\end{symbollist}

\begin{figure}
\includegraphics[width=\columnwidth]{figures/Fig5.png}
\caption{Schematic maximum bed level changes at the upstream end.}
\label{Fig5}
\end{figure}

\autoref{Eq7} implies that at the end of period $i$ (lasting for a period of $T_i$ days) the morphological effect of the measure is given by
%
\begin{equation}
z_{b,i+1}(0) = z_{b,i} (0) \sigma_i + z_{b,i,\text{eq}} (1-\sigma_i) \text{ with } \sigma_i = e^{-T_i/T_{m,i}}
\label{Eq8a}
\end{equation}

If the yearly hydrograph is schematized using three periods, one obtains the following set of three equations for the three periods by assuming periodicity (the final bed level of the third period equals the initial bed level of the first period)
%
\begin{align}
z_{b,2}(0) &= z_{b,1}(0) \sigma_1 + z_{b,1,\text{eq}} (1-\sigma_1) \label{Eq8b} \\
z_{b,3}(0) &= z_{b,2}(0) \sigma_2 + z_{b,2,\text{eq}} (1-\sigma_2) \label{Eq8c} \\
z_{b,1}(0) &= z_{b,3}(0) \sigma_3 + z_{b,3,\text{eq}} (1-\sigma_3) \label{Eq8d}
\end{align}

We can solve these three equations for the unknown bed levels $z_{b,1}(0)$, $z_{b,2}(0)$ and $z_{b,3}(0)$.
This gives the following three expressions
%
\begin{align}
z_{b,1}(0) &= \frac{z_{b,1,\text{eq}} (1-\sigma_1) \sigma_2 \sigma_3 + z_{b,2,\text{eq}} (1-\sigma_2) \sigma_3 + z_{b,3,\text{eq}} (1-\sigma_3)}{1 - \sigma_1 \sigma_2 \sigma_3} \label{Eq8e} \\
z_{b,2}(0) &= \frac{z_{b,1,\text{eq}} (1-\sigma_1) + z_{b,2,\text{eq}} (1-\sigma_2) \sigma_3 \sigma_1 + z_{b,3,\text{eq}} (1-\sigma_3) \sigma_1}{1 - \sigma_1 \sigma_2 \sigma_3} \label{Eq8f} \\
z_{b,3}(0) &= \frac{z_{b,1,\text{eq}} (1-\sigma_1) \sigma_2 + z_{b,2,\text{eq}} (1-\sigma_2) + z_{b,3,\text{eq}} (1-\sigma_3) \sigma_1 \sigma_2}{1 - \sigma_1 \sigma_2 \sigma_3} \label{Eq8g}
\end{align}

The lower bound for the morphological impact of the measure will be equal to $\min[z_{b,1}(0), z_{b,2}(0), z_{b,3}(0)]$ and the upper bound is given by $\max[z_{b,1}(0), z_{b,2}(0), z_{b,3}(0)]$.
For the schematized hydrograph shown in \autoref{Fig5} the bed level change is

\begin{itemize}
\item maximum at $z_{b,1}(0)$ (\autoref{Eq8e}) after flood period 3 (the third block in \autoref{Tab2})
\item minimum at $z_{b,2}(0)$ (\autoref{Eq8f}) after the low flow period 1 (the first block in \autoref{Tab2}).
\end{itemize}

In case the river section is experiences nearly stagnant flow conditions due to barriers a yearly period without bed level changes is inserted between low flow period 1 and the transitional flow period 2.\footnote{Earlier versions of this manual suggested that this period might be inserted between the flood period 3 and the low flow period 1, but algorithmically it was always placed between periods 1 and 2.}

We can't only obtain expressions for the two extreme (minimum, maximum) values but we can also determine an expression for the year-averaged bed level change by integrating \autoref{Eq7} per constant discharge period to
%
\begin{align}
\bar{z}_{b,i} &= \frac{1}{T_i} \int_0^{T_i}{z_{b,i} (0) + [z_{b,i,\text{eq}} - z_{b,i}(0)](1 - e^{-t/T_{m,i}})}dt \\
&= \frac{1}{T_i} \left . \left ( {z_{b,i,\text{eq}} t + [z_{b,i,\text{eq}} - z_{b,i}(0)]T_{m,i} e^{-t/T_{m,i}}} \right ) \right |_{t=0}^{t=T_i} \\
&= z_{b,i,\text{eq}} + [z_{b,i,\text{eq}} - z_{b,i}(0)] \frac{T_{m,i}}{T_i} ( e^{-T_i/T_{m,i}} - 1 )
\end{align}
%
and over all periods to
%
\begin{equation}
\bar{z}_b = \frac{\sum{z_{b,i,\text{eq}} T_i}}{\sum{T_i}} + \frac{\sum{(z_{b,i,\text{eq}}-z_{b,i}(0)) T_{m,i} (\sigma_i-1)}}{\sum{T_i}}
\label{Eq8h}
\end{equation}
%
If all bed level changes are removed after the flood period by means of dredging, the yearly dredging amount can finally be estimated by ignoring any excess depth.
After all, the maximum bed level change at the end of the flood season can with $z_{b,1}(0) = 0$ and \autoref{Eq8a} to \autoref{Eq8i} be expressed as the maximum dredging depth $z_\text{mdd}$

\begin{equation}
z_\text{mdd} = z_{b,1,\text{eq}}(1-\sigma_1) \sigma_2 \sigma_3 + z_{b,2,\text{eq}} (1-\sigma_2) \sigma_3 + z_{b,3,\text{eq}} (1-\sigma_3)
\label{Eq8i}
\end{equation}

Because this estimated doesn't take into account the sediment supply by the river, \autoref{Eq8i} can overestimate the maintenance dredging required.
\autoref{Eq8i} is therefore not included in the \dfastmi analysis.

\section{Estimate of the spatial distribution of the bed level changes}

\autoref{Eq8e} gives the maximum and \autoref{Eq8f} gives the minimum bed level change at the upstream end of the measure.
Moving downstream for that point, the bed level change consists of a minimum bed level change $z_{b,1,\text{eq}}$ plus the part of the flood deposit that moves downstream during low flow conditions.
During a flood the downstream migrating sedimentation volume may temporarily even cause bed level changes larger than $z_{b,2,\text{eq}}$ but this is rather unlikely.
For convenience, it's therefore assumed that the maximum and minimum bed level change given by \autoref{Eq8e} and \autoref{Eq8f} are valid \emph{for each individual point in the main channel within the impacted area}.

This approximation implies that first the values of $z_{b,1,\text{eq}}$, $z_{b,2,\text{eq}}$ and $z_{b,3,\text{eq}}$ can be determined for every computational cell of the simulation in the main channel.
Subsequently the minimum (at the end of the low flow period) and maximum bed level change (at the end of the flood period) can be estimated given the approximated values for $\sigma_1$, $\sigma_2$ and $\sigma_3$ and \autoref{Eq8f} and \autoref{Eq8e} respectively.

\section{Time scales for bed level change}

The magnitude of the maximum bed level change at the end of the flood period and the minimum bed level change at the end of the low flow period both depend on the morphological time scales $T_{m,i}$ \unitbrackets{day} which define the rate of response of the bed levels.
Using \autoref{Eq5T} the time scale $T_{m,i}$ can also be written as $T_{m,i} = L/w_i$ with $w_i$ the bed celerity, i.e.~the propagation speed of bed level changes, and $L$ the distance measure along the flow direction over which the bed level changes are accumulated.
As mentioned before, the length $L$ corresponds to the distance over which flow leaves the main channel and the flow pattern in the main channel adjusts to the reduced discharge.
Obviously, this distance varies in reality over the channel width and it depends on the discharge condition considered.
For consistent use of the rule-of-thumb, it's assumed that the length $L$ corresponds to twice the main channel width $B_\text{mc}$.

The second parameter in the definition of the morphological time scale is the bed celerity.
Based on statistics of width-averaged bed level observations averaged per km chainage year-averaged values have been determined \citep{RIZA2005} for the Dutch rivers.
These values are presented in \autoref{Tab2} and \autoref{Tab3} for Rhine branches and Meuse respectively.

\begin{table}
\includegraphics[width=\columnwidth]{figures/Tab2.png}
\caption{Overview of average bed celerities (based on km-averaged bed levels including the effects of dredging) by \citet{RIZA2005}.}
\label{Tab2Again}
\end{table}

\begin{table}
\includegraphics[width=\columnwidth]{figures/Tab3.png}
\caption{Overview of the reach averaged bed celerities (based on km-averaged bed levels including the effects of dredging) by \citet{Waterdienst2008}.}
\label{Tab3Again}
\end{table}

\begin{table}
\includegraphics[width=\columnwidth]{figures/Tab4.png}
\caption{Overview of the reach averaged bed celerities (based on km-averaged bed levels for the period 1975-2000 including the effects of dredging) by \citet{RIZA2007}.}
\label{Tab4}
\end{table}

In order to translate these empirical year averaged values to discrete values per discharge block the following approximation is used for free flowing Rhine branches.
For the Bovenrijn discharges which are predominantly larger than 4000 m\textsuperscript{3}/s (during on average 8 \% of the year) a "high flow bed celerity" $w_h$ of 10 m/dag (3.65 km/yr) is used.

For discharge blocks which are predominantly below 4000 m\textsuperscript{3}/s a "low flow bed celerity" $w_l$ is used which is derived from the year-averaged bed celerity $\bar{c}$ as

\begin{equation}
w_l = \frac{\bar{c} - 0.082 \cdot 365}{0.918}
\label{Eq9}
\end{equation}

However, this approximation \autoref{Eq9} is only valid for the free flowing Rhine branches, hence it's for instance not valid for the Nederrijn where at low discharges the barriers are closed (\autoref{Fig6}).

\begin{figure}
\includegraphics[width=\columnwidth]{figures/Fig6.png}
\caption{Nederrijn discharges at Driel-boven based on Donar database for the period 2000-2004).}
\label{Fig6}
\end{figure}

\autoref{Fig6} shows that the barriers in the Nederrijn are opened at a discharge of 1500 m\textsuperscript{3}/s in the Bovenrijn at Lobith (this value is exceeded during 57 \% of the year) and that the branch is approximately free flowing for discharges above 2200 m\textsuperscript{3}/s in the Bovenrijn at Lobith (exceeded during 33 \% of the year).
It's therefore assumed that the year-averaged bed celerity related to the low and high flow values as $\bar{c} = 0.082 w_h + (0.918 - 0.33) w_l$ such that for $w_h = 3,65$ km/yr we obtain

\begin{equation}
w_l = \frac{\bar{c} - 0.082 \cdot 3.65}{0.918 - 0.33}
\label{Eq10a}
\end{equation}

For the Meuse the we apply the following approximation.
It's assumed that the Meuse barriers are open for on average 10 days per year (3 \% of the time) such that the bed celerity can be estimated as

\begin{equation}
w_h = \frac{\bar{c}}{0,03}
\label{10b}
\end{equation}


Finally, due to the gradation the river bed of the Grensmaas will only become mobile at discharges above 1000 m\textsuperscript{3}/s which occurs on average about 4 days a year.
The bed celerity is thus estimated as

\begin{equation}
w_h = \frac{\bar{c}}{0,01}
\label{Eq10c}
\end{equation}

Based on the approximations \autoref{Eq9} to \autoref{Eq10c} and the year-averaged bed celerity values in \autoref{Tab2Again} to \autoref{Tab4} we obtain estimates for the bed celerities during high and low flow conditions.

\begin{table}
\includegraphics[width=\columnwidth]{figures/Tab4_the2nd.png}
\caption{Representative bed celerities during high- and low-flow conditions Rhine branches.}
\label{Tab4RT}
\end{table}

\begin{table}
\includegraphics[width=\columnwidth]{figures/Tab5.png}
\caption{Representative bed celerities during high- and low-flow conditions Meuse.}
\label{Tab5}
\end{table}
