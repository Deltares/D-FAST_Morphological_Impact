\chapter{Introduction}
In 2020 the rapid assessment tool \dfastmi (MI) was developed for the 6th generation models based on \dflowfm.
The algorithms implemented in \dfmi version 2 correspond to the algorithms implemented in the older equivalent tool WAQMORF for the SIMONA Suite.
In 2021 a number of extensions to the \dfastmi algorithms were implemented in a pilot version and evaluated.
\dfmi version 3, the current release is based on the first of these extensions, namely the stepped-hydrograph for the reaches with dominant river discharge forcing.

This document describes the cases used to validate this version.
It is subdivided into three chapters:

\begin{itemize}
\item \nameref{Chp:Backward}. This chapter verifies that results of \dfmi 3, when running in the classic 3-discharge backward compatibility mode, are identical to the results obtained using \dfmi 2 and WAQMORF.
This validates that the results of the backward compatibility mode are correct, as the old results have already been accepted as realistic in previous studies.
\item \nameref{Chp:Morphology}. This chapter compares the results of \dfmi 3, when using the new stepped-hydrograph mode, with the results of actual long-term morphological simulations.
This validates that the results of the new stepped-hydrograph mode agree with those obtained using long-term morphological simulations for a number of reference interventions (note that this does not guarantee that results will agree for all possible interventions). 
\item \nameref{Chp:Verschil}. This chapter compares the results of \dfmi 3, when using the new stepped-hydrograph mode, with the results obtained using the classic 3-discharge approach.
This verifies that the results of the new stepped-hydrograph approach do not deviate substantially from results obtained using the old method.
Note that the old results were accepted as realistic in previous studies.
\end{itemize}

All analyses have been carried using results obtained from either Delft3D 4 or WAQUA.
However, \dfmi 3 is intended for use with results obtained from \dflowfm.
During the initial development of \dfmi 2 it was verified that the 3-discharge method gave similar results when using WAQUA results and when using \dflowfm results \citep{DFAST2020}.
This is as expected since both models have been calibrated in a similar way against the same data sets.
Therefore, it is expected that differences will be mainly observed when studying small interventions that are barely resolved on the computational mesh.
This conclusion also holds for the new stepped-hydrograph method implemented in \dfastmi 3.