\chapter{File formats}

The software distinguishes 6 files:

\begin{itemize}
\item The \emph{rivers configuration file} defines the branches and reaches, and all parameter settings specific for the overall system, per branch or per reach.
\item The \emph{dialog text file} defines all strings to be used in the interaction with the users (GUI, report, or error messages).
\item The \emph{analysis configuration file} defines the settings that are relevant for a specific execution of the algorithm, i.e.~for a specific branch, reach and intervention.
\item The \emph{simulation result files} define the spatial variations in the velocities and water depths as needed by the algorithm.
\item The \emph{report file} contains a logging of the settings and lumped results for the analysis.
\item The \emph{spatial output file} contains the estimate of the spatial variation in the sedimentation and erosion patterns that will result from the intervention (minimum, mean and maximum).
\end{itemize}

Each file type is addressed separately in the following subsections.

\section{Rivers configuration file}\label{sec:rivers-config}

The rivers configuration file follows the common ini-file format.
The file must contain a \keyw{[General]} block with a keyword \keyw{Version} to indicate the version number of the file.
The version number should be \keyw{3.0}.\footnote{Initially this file version was named \keyw{2.0} but to avoid confusion with the fact that this file format is only supported by \dfastmi version 3.0, the version number has been increased.
Files with version \keyw{2.0} are still accepted, but not recommended.}

Besides the \keyw{[General]} block version 3.0 files should only contain data blocks identifying the river branches (in Dutch: takken).
The names of those blocks will be used as branch names.
For instance, a block [My branch] will be processed as a branch called "My branch".
The order of the branches in the gui will correspond to the order of the blocks in the file.
The branch block defines the reaches (in Dutch: stukken) to be distinguished as well as the branch or reach specific parameter settings.
The \keyw{[General]} block may contain default values for the various parameters.
Further details follow below.

\begin{longtable}{l|l|p{8cm}}
\caption{Description of rivers configuration version 3} \\
Block & Keyword & Description \\ \hline
\endfirsthead
\multicolumn{3}{l}{\textsl{(continued from previous page)}} \\
Block & Keyword & Description \\ \hline
\endhead
\hline \multicolumn{3}{r}{\textsl{(continued on next page)}} \\
\endfoot
\endlastfoot
\keyw{General} & \keyw{Version} & Version number.
Should be \keyw{3.0}. \\
\keyw{General} & \keyw{checksum} & Checksum of the rivers configuration file.
Used to verify that the rivers configuration file wasn't accidentally modified. \\
BranchName<i> & \keyw{QLocation} & Location at which discharges for branch <i> are defined. \\
BranchName<i> & \keyw{Reach<j>} & Name of reach <j> within branch <i>. \\
\keyw{*} & \keyw{NWidth} & Normal width \unitbrackets{m} of main channel. \\
\keyw{*} & \keyw{QStagnant} & Discharge \unitbrackets{m\textsuperscript{3}/s} below which main channel flow can be assumed stagnant. \\
\keyw{*} & \keyw{UCrit} & Critical (minimum) velocity \unitbrackets{m/s} for sediment transport. \\

\keyw{*} & \keyw{HydroQ} & Series of discharges \unitbrackets{\SI{}{\metre\cubed\per\second}} representing each stage of the ``hydrograph''. \\
\keyw{*} & \keyw{Tide} & Flag to indicate whether the stages of the ``hydrograph'' are represented by discharge only (False) or by a combination of discharge and tide (True). \\
\keyw{*} & \keyw{TideBC} & Series of other tidal conditions representing the stages of the ``hydrograph''.
Only used in case \keyw{Tide} equals True. \\

\keyw{*} & \keyw{AutoTime} & Flag to indicate whether the duration of each discharge period should automatically be derived from the exceedance curve given by \keyw{QFit}.
The default value is False. \\
\keyw{*} & \keyw{HydroT} & Series of values specifying the duration of each stage of the ``hydrograph''.
The unit used for specifying the durations is arbitrary: the sum of all durations is assumed to be equivalent to 1 year.
The length of the series should be equal to the length of the series of discharges specified for \keyw{HydroQ}.
Only used in case \keyw{AutoTime} equals False. \\
\keyw{*} & \keyw{QFit} & Two discharges \unitbrackets{\SI{}{\metre\cubed\per\second}} used for representing the exceedance curve in case \keyw{AutoTime} equals True. \\

\keyw{*} & \keyw{CelerForm} & Flag to indicate the method used for determining the bed celerity.
The options are 1 (series of discharge, bed celerity pairs) and 2 (power-law relation between celerity and discharge).
The default value is 2. \\
\keyw{*} & \keyw{PropQ} & Strictly-monotonous increasing list of discharges \unitbrackets{\SI{}{\metre\cubed\per\second}} for which the bed celerity is given.
Only used in case \keyw{CelerForm} equals 1. \\
\keyw{*} & \keyw{PropC} & List of bed celerities \unitbrackets{\SI{}{\kilo\metre\per\year}} one value for each discharge specified by \keyw{PropQ}.
The length of the lists given by \keyw{PropQ} and \keyw{PropC} should be equal.
Only used in case \keyw{CelerForm} equals 1. \\
\keyw{*} & \keyw{CelerQ} & Two values representing the multiplication scalar and exponent of the power-law relation of the bed celerity \unitbrackets{\SI{}{\kilo\metre\per\year}} as function of the discharge \unitbrackets{\SI{}{\metre\cubed\per\second}}.
Only used in case \keyw{CelerForm} equals 2.
\end{longtable}

All keywords listed for block \keyw{*} may occur either in the \keyw{[General]} block or in one of the branch specific blocks where they may optionally be concatenated with the reach number <j>.
Those keywords may thus occur in three flavours:

\begin{enumerate}
\item Plain keyword in block \keyw{[General]}: global default value valid for all branches
\item Plain keyword in branch specific block: default value for that branch (overrules any global default)
\item Keyword followed by reach number <j> in branch specific block: value valid for that specific reach (on that branch).
\end{enumerate}

\subsubsection*{Example}

The following excerpt of the default \keyw{Dutch\_rivers.ini} configuration file shows the \keyw{[General]} as well as the first part of the \keyw{[Bovenrijn \& Waal]} block for the first branch.
It includes a global value of 0.3 for \keyw{UCrit} and 100 for \keyw{QMin}.
The other parameters are specified at branch level and mostly uniform for the whole branch.
Only \keyw{NWidth} and \keyw{PRLow} vary depending on the reach selected.

\begin{Verbatim}
    [General]
        Version    = 3.0
        UCrit      = 0.3
        CelerForm  = 1
        checksum   = 2377805458
    
    [Bovenrijn & Waal]
        QLocation  = Lobith
        HydroQ     = 1300 2000 3000 4000 6000 8000
        AutoTime   = True
        QStagnant  = 800
        QFit       = 800  1280
    
        Reach1     = Bovenrijn                    km  859-867
        NWidth1    = 340
        PropQ1     = 3500  3501
        PropC1     = 0.89  3.65
    
        Reach2     = Boven-Waal                   km  868-886
        NWidth2    = 260
        PropQ2     = 3500  3501
        PropC2     = 0.81  3.65

        ... continued ...
\end{Verbatim}

\section{Dialog text file}

The dialog text file uses the block labels enclosed by square brackets of the common ini-file format, but the lines in between the blocks are treated verbatim and don't list keyword/value pairs.
Every print statement in the program is associated with a short descriptive identifier.
These identifiers show up in the dialog text file as the block labels.
The text that follows the block label will be used at that location in the program.
The order of the blocks in the file is not important.
Please note that every line is used as is, so don't add indentations or blank lines unless you want those to show up during the program execution.
Most blocks may contain any number of lines, but some blocks may only contain a single line in particular block that start with \keyw{gui\_} or \keyw{filename\_}.
Some data blocks may contain one or more named placeholders, e.g. \keyw{{version}}, used for inserting values by means of the Python \keyw{format()} method.

\subsubsection*{Example}

The following excerpt of the default \keyw{messages.NL.cfg} dialog text file shows the string definition for 5 identifiers, namely '' (the identifier for an empty line), 'header', 'confirm', 'confirm\_or' and 'confirm\_or\_restart'.
The header string contains one placeholder, namely \keyw{{version}} for the the version number.

\begin{Verbatim}
    []
    
    [reduce_output]
    The option 'reduce_output' is active.
    
    [header]
    D-FAST Morphological Impact implements an algorithm to estimate the local
    morphological effects of a local intervention (i.e. an adjustment to the
    river). The conceptual framework was originally introduced in
        "RWS-WD memo WAQUA vuistregel (Sieben, 2010)"
    but it has been extended and improved over the years. Check the user manual
    for the details of the currently implemented algorithm.
    
    It is based on an estimation of the equilibrium bed level changes in the
    main channel that would occur eventually when river maintenance would not
    be adjusted.
    
    The effect is expressed as:
    
        year-averaged bed level change [m] without dredging
        maximum bed level change [m] without dredging
        minimum bed level change [m] without dredging
    
    By means of these estimates bottlenecks can be identified. The results are
    not suitable for direct estimation of the impact on the maintenance of the
    navigation channel!
    
    The combination of the total equilibrium sedimentation volume and the
    yearly sediment load of the river determines the period over which the
    equilibrium can be reached.
    
    This is version {version}.
    
    [confirm]
    Confirm using "y" ...
    
    [confirm_or]
    Confirm using "y", or reply "n" ...
    
    ... continued ...
\end{Verbatim}

\section{Analysis configuration file}\label{app:config}

The analysis configuration file follows the common ini-file format.
The file must contain a \keyw{[General]} block with a keyword \keyw{Version} to indicate the version number of the file.
The version number should be \keyw{3.0}.\footnote{Initially this file version was named \keyw{2.0} but to avoid confusion with the fact that this file format is only supported by \dfastmi version 3.0, the version number has been increased.
Files with version \keyw{2.0} are still accepted, but not recommended.}

Version 3.0 files must contain in the \keyw{[General]} block also the keywords \keyw{Branch} and \keyw{Reach} to identify the branch (in Dutch: tak) and reach (in Dutch: stuk) in which the intervention is located.
The specified names may be shortened, but they should uniquely identify the branch and reach amongst the names of the other branches and reaches.
The same block may also contain \keyw{QThreshold} and \keyw{UCrit} values representative for this particular intervention if they differ from those typical for the selected reach.
Furthermore, this block may contain \keyw{FigureDir} and \keyw{OutputDir} specifying where the figures and other output files should be written.
The \keyw{RiverKM} keyword to specify the chainage along the reach of interest is needed for estimating the initial year dredging volumes.
Last but not least, the user needs to specify the names of the D-Flow FM map- or fourier-files containing the results of the simulations without intervention (reference) and with intervention for the selected flow conditions.
These names must be specified in a continuous sequences of numbered blocks named \keyw{C1}, \keyw{C2}, etc.
The order of the blocks is not important: the relevant blocks are identified by means of the \keyw{Discharge} and optionally \keyw{TideBC} specified per block.
There should be a match for every stage of the configured ``hydrograph'' for the selected branch/reach.
The file names may be specified using relative or absolute paths.

\begin{tabular}{l|l|p{8cm}}
Block & Keyword & Description \\ \hline
\keyw{General} & \keyw{Version} & Version number.
Should be \keyw{3.0}. \\
\keyw{General} & \keyw{CaseDescription} & Description of the analysis. \\
\keyw{General} & \keyw{Branch} & Name of the selected branch. \\
\keyw{General} & \keyw{Reach} & Name of the selected reach. \\
\keyw{General} & \keyw{QThreshold} & Minimum discharge \unitbrackets{\SI{}{\metre\cubed\per\second}} at which intervention becomes active. \\
\keyw{General} & \keyw{UCrit} & Critical (minimum) velocity \unitbrackets{\SI{}{\metre\per\second}} for sediment transport. \\
\keyw{General} & \keyw{RiverKM} & Name of file with river chainage \unitbrackets{\SI{}{\kilo\metre}} and corresponding xy-coordinates. \\
\keyw{General} & \keyw{FigureDir} & Directory for storing figures (default relative to work dir: figure). \\
\keyw{General} & \keyw{OutputDir} & Directory for storing output files. \\
\keyw{C}<i> & \keyw{Discharge} & Discharge \unitbrackets{m\textsuperscript{3}/s} of condition <i>. \\
\keyw{C}<i> & \keyw{TideBC} & Tidal boundary of condition <i>. \\
\keyw{C}<i> & \keyw{Reference} & Name of D-Flow FM map- or fourier-file to be used for reference condition <i>. \\
\keyw{C}<i> & \keyw{WithIntervention} & Name of D-Flow FM map- or fourier-file that includes the intervention <i>. \\
\end{tabular}

\subsubsection*{Example}

This example shows a complete analysis configuration file for an intervention in the first branch/reach of the default \keyw{Dutch\_rivers.cfg} configuration.
It reports the default settings.
Only the \keyw{Version}, \keyw{Branch}, \keyw{Reach}, \keyw{Reference} and \keyw{WithIntervention} keywords are required for the full analysis.

\begin{Verbatim}
    [General]
      Version          = 3.0
      CaseDescription  = 
      Branch           = Bovenrijn & Waal
      Reach            = Boven-Waal                   km  868-886
      Qthreshold       = 1000.0
      Ucrit            = 0.3
      OutputDir        = 
      Plotting         = False
      SavePlots        = False
      FigureDir        = 
      ClosePlots       = False
      RiverKM          = 
    
    [C1]
      Discharge        = 1300.0
      Reference        = Case42/Q1/Reference/DFM_OUTPUT_Q1/Q1_map.nc
      WithIntervention = Case42/Q1/Updated/DFM_OUTPUT_Q1/Q1_map.nc
    
    [C2]
      Discharge        = 2000.0
      Reference        = Case42/Q2/Reference/DFM_OUTPUT_Q2/Q2_map.nc
      WithIntervention = Case42/Q2/Updated/DFM_OUTPUT_Q2/Q2_map.nc
    
    [C3]
      Discharge        = 3000.0
      Reference        = Case42/Q3/Reference/DFM_OUTPUT_Q3/Q3_map.nc
      WithIntervention = Case42/Q3/Updated/DFM_OUTPUT_Q3/Q3_map.nc

        ... continued ...
\end{Verbatim}

\section{Simulation result files}

\dfastmi expects all data in UGRID netCDF file similar to the map-file output of \dflowfm.
\dfastmi reads the results directly no conversion is necessary.
These files may contain multiple time steps; the final time steps will be used for the analysis.
The mesh geometry is transferred from one of the simulation files to the \dfastmi spatial output "results" file.

\section{Report file}

\dfastmi will write a report of the analysis.
This report file is a simple text file consistent with the earlier reports written by WAQMORF.
The length and content of the report vary depending on the availability of the simulation result files and the language selected.


\section{Spatial output file}

\dfastmi generates one UGRID netCDF file containing the spatial results of the analysis.
The mesh information is copied from the first D-Flow FM map- or fourier-file read, and the three data fields (erosion and sedimentation patterns for mean, minimum, and maximum impact) follow standardized conventions for data stored at cell centres (\keyw{face}-values) on an unstructured mesh.
As a result the data may be visualized using a number of visualization tools such as QUICKPLOT and QGIS.
