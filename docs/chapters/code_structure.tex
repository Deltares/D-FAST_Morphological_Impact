\chapter{Code structure}

\section{Introduction}

\dfastmi code is subdivided into 6 files:

\begin{itemize}
\item \keyw{\_\_init\_\_.py} module level file containing mainly the version number.
\item \keyw{\_\_main\_\_.py} module level file containing argument parsing and call to \keyw{dfastmi.cmd.run()}.
\item \keyw{cmd.py} containing the main run routine.
\item \keyw{cli.py} for the backward compatible interactive command line interface (equivalent to WAQMORF).
\item \keyw{plotting.py} contains all plotting related functionality (most plots are related to dredging volume estimates and for research purposes only).
\item \keyw{resources.py} is used to link to the \dfastmi logo.
\end{itemize}

and 4 sub-packages:

\begin{itemize}
\item \keyw{gui} contains all functionality related to the graphical user interface.
\item \keyw{batch} implements the workflow for the actual \dfastmi analysis (triggered either by a \keyw{-{}-mode batch} call or by clicking the \button{Compute} button in the graphical user interface).
\item \keyw{io} contains the reading, parsing and writing of all configuration and result files.
\item \keyw{kernel} contains the scientific steps of the analysis.
\end{itemize}

The further description of the code in the initial technical reference manual was largely obsolete due to the restructuring of the code for \dfastmi 3.0.
A more detailed description of the functionality may be added in a future revision of this document.