\chapter{Developer Guide}

\section{Introduction}

This chapter gives an overview of how \dfastmi can be setup for development.

The development setup and configuration is done for usage of the visual studio code IDE. More information about visual studio code can be found on the internet.
The development of python code project within Deltares is mostly done in with this tool. Visual Studio Code is free for private or commercial use. See the \href{https://code.visualstudio.com/license}{product license}) for details.

\section{Configuration}
The tool can be configured with a batch script which will configure using installed tools or ask the user to download and install the tools which can be used for them. The batch file is located in the sub folder \textbf{BuildScripts} and is called \textbf{DevelopDfastmi.bat}.

The batch file will check for installed tooling for anaconda/miniconda. If not installed it will ask the user to install the anaconda/miniconda tooling for you. \textbf{PLEASE NOTE:} you need to close the command line and re-open / restart the batch script as it needs to initialize the conda tooling for the command line and powershell (used in VSCode).

The batch file will check for installed tooling for VSCode. If not installed it will ask the user to install the VSCode tooling for you. \textbf{PLEASE NOTE:} you need to close the command line and re-open / restart the batch script as it needs to initialize the conda tooling for the command line and powershell (used in VSCode).

\subsection{Conda environment}
We setup a conda environment on the client pc so we have the python interpreter we expect to be used by developers.

\section{Tooling}
The tool is using:
\begin{enumerate}
\item \keyw{anaconda} / \keyw{miniconda} /\keyw{conda} to setup its python environment. 
\item \keyw{poetry} to install the python packages for the environment
\item \keyw{VSCode} to edit python files, run the created (unit)tests, see/visualize the test coverage
\end{enumerate}

\subsection{Utility / install scripts}
We use several other batch scripts.
\begin{itemize}
	\item \keyw{CondaInstall.bat} used to install miniconda on the client pc, you need to restart the main script (DevelopDfastmi.bat) in a \textbf{new} command line prompt because the command line prompt environment is
	 updated after install.
	\item \keyw{VSCodeInstall.bat} used to install visual studio code on the client pc, you need to restart the main script (DevelopDfastmi.bat) in a \textbf{new} command line prompt because the command line prompt environment is updated after install.
\end{itemize}

\subsection{VSCode Extensions}
In VSCode we use extensions to find python unit test, visualize coverage and debug our code. To do this we install the following extensions.
\begin{enumerate}
	\item \keyw{Cameron.vscode-pytest}
	\item \keyw{donjayamanne.python-environment-manager}
	\item \keyw{hbenl.vscode-test-explorer}
	\item \keyw{littlefoxteam.vscode-python-test-adapter}
	\item \keyw{ms-python.pylint}
	\item \keyw{ms-python.python}
	\item \keyw{ms-python.vscode-pylance}
	\item \keyw{ms-vscode.test-adapter-converter}
	\item \keyw{ryanluker.vscode-coverage-gutters}
\end{enumerate}

\subsection{VSCode userfiles}
The user files can be found in subfolder \textbf{.vscode}
\begin{enumerate}
	\item \keyw{extensions.json} 
	\item \keyw{launch.json} 
	\item \keyw{settings.json} 
\end{enumerate}

