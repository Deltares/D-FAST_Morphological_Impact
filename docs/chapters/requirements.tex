\chapter{Software requirements}

\section{Introduction}

The purpose of \dfastmi is to provide a first estimate of the bed level changes in the main channel if local measures were to be implemented outside the main channel.
It is based on the conceptual WAQMORF framework originally developed by \citet{Sieben2008}.
WAQMORF was developed for the SIMONA system.
\dfastmi implements the same functionality based on results obtained from D-Flow Flexible Mesh.
The detailed list of requirements is given below.

\section{Functional Requirements} \label{Sec:FuncReq}

\begin{enumerate}
\item This program must give the same results for the same data as WAQMORF.
\item Users must be able to run this program in batch mode from the command line.
\item Users must be able to run the analysis based on \dflowfm results.
\item Users must be able to provide all data via an input file.
\item The input file must be easy to edit for users, i.e.~a text file.
\item The report output must be a simple text file consistent with WAQMORF.
\item The spatial output must be easy to visualize in common software.

\item The program should read relevant data directly from \dflowfm map-files instead of intermediate XYZ files as required by WAQMORF for SIMONA results.

\item The input file could use the ini-format consistent with \dflowfm input files.
\item A simple graphical user interface could support users in process of creating the input file.

\item It would be nice if the software would be more generally applicable than just the Dutch rivers.
\item It would be nice if the software would be able to run besides Dutch also in English.
\end{enumerate}

All requirements are addressed by \dfastmi.

\section{Non-functional requirements}

\begin{enumerate}
\item The performance of the tools must be similar to that of WAQMORF, i.e.~it must run within seconds.
\item The software must be version controlled.
\item The software must have formal testing and support.
\item The software must run on Windows.
\item The software must be easy to distribute.
\item The software must have a user manual.
\item The software must have technical documentation.

\item The software should run on any common operating system.
\item The software should be available as open source.
\end{enumerate}

All requirements are addressed by \dfastmi although the testing has been carried out on the Windows platform only.
